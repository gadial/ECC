\documentclass[11pt,english]{article}
\usepackage[latin9]{inputenc}
\usepackage{geometry}
\usepackage{subfig}	%%for easy placement of multiple pictures inside one figure
\geometry{verbose,letterpaper,tmargin=1in,bmargin=1in,lmargin=1.3in,rmargin=1.3in}
\usepackage{graphicx}
\usepackage{amssymb}
\makeatletter

%%%%%%%%%%%%%%%%%%%%%%%%%%%%%% LyX specific LaTeX commands.
\newcommand{\noun}[1]{\textsc{#1}}
%% Bold symbol macro for standard LaTeX users
\providecommand{\boldsymbol}[1]{\mbox{\boldmath $#1$}}

%%%%%%%%%%%%%%%%%%%%%%%%%%%%%% User specified LaTeX commands.
\usepackage{color,calc}
\definecolor{shade}{gray}{0.984}
\newenvironment{myFramedBox}[1][]%
        {
        %\setlength{\fboxsep}{-\fboxrule}
        % \footnotesize\normalfont\ttfamily\raggedright
        \setlength{\rightmargin}{\leftmargin}
        \setlength{\itemsep}{-12pt}
        \setlength{\parsep}{20pt}
        \begin{lrbox}{\@tempboxa}%
        \begin{minipage}{\linewidth-2\fboxsep}
        }%
        {
        \end{minipage}%
        \end{lrbox}%
        \fcolorbox{black}{shade}{\usebox{\@tempboxa}}\newline\newline
        }%

\usepackage{babel}
\makeatother

\setlength{\textwidth}{6.50in}      % was 6.00
\setlength{\evensidemargin}{0.05in} % was 0.25
\setlength{\oddsidemargin}{0.05in}  % was 0.25
\setlength{\textheight}{8.80in}     % was 8.5 or 9.0
\setlength{\topmargin}{-0.7in}      % was -0.5
\setlength{\parskip}{2.0mm}         % was 2
\setlength{\baselineskip}{1.7\baselineskip}

\newtheorem{theorem}{Theorem}
\newtheorem{lemma}[theorem]{Lemma}
\newenvironment{proof}{{\noindent \em Proof:~}}{\hfill{\hfill$\Box$}}

\newcommand{\complexityclass}[1]{{\bf #1}}
\newcommand{\NP}{\complexityclass{NP}}
\newcommand{\RE}{\complexityclass{RE}}

\begin{document}

\def\thepage{}

\title{Elliptic Curve Cryptography\\Final report for a project in computer security}


\author{
   Gadi Aleksandrowicz\thanks{
      Dept.\ of Computer Science,
      The Technion---Israel Institute of Technology,
      Haifa~32000, Israel.
      E-mail: {\tt gadial@cs.technion.ac.il}
   } \and
   Basil Hess
}

\date{}

\maketitle

\section{Introduction}
An \emph{Elliptic Curve} can be roughly described as the set of solutions to an equation of the form $y^2=x^3+ax+b$ over some field (e.g. $\mathbb{C}, 
\mathbb{R},\mathbb{Q}$ or some finite field $\mathbb{F}_{p^n}$). The importence of elliptic curves stems from their rich structure: there is a rather simple
addition law definable on elliptic curves which makes them into an abelian group. Studying the emerging structure of elliptic curves over various fields has been a
major theme in the mathematics of the 20th centaury, and elliptic curves were connected to many famous problems and results, most notably the proof of Fermat's
last theorem\marginpar{Give some reference, to a good survey if possible}.

In the late 70's, \emph{Public-Key Cryptography} systems were first publicly described, changing the face of cryptography. Many of the purposed systems, such
as the Diffie-Hellman key exchange system and the ElGamal encryption system\marginpar{Reference to both papers}, were based on arithmetic in the group $\mathbb{Z}_p$, but in theory could
be implemented in other groups as well. Groups based on elliptic curves were a good choice because of their well-developed theory and their high variety. In 1985,
both Neal Koblitz and Victor S. Miller \marginpar{Ref...} suggested public key cryptosystems based on elliptic curves.

Our goal in the project has been to implement an efficient public key cryptosystem based on elliptic curves ``from scratch'', relying only on large-number arithmetic
libraries. In particular, we have implemented all the elliptic-curve related calculations, and additional related algorithms.
\end{document}
